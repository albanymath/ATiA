% !TeX program = xelatex
\documentclass{UAmathtalk}

\author{Sarah Tymochko}
\address{Michigan State University}
\urladdr{https://www.egr.msu.edu/~tymochko/}
\title{Using Zigzag Persistence for Bifurcation Detection}
\date{Friday, October 15th, 2021}


\begin{document}

\maketitle

%\begin{center}
%\Large{Workshop in Topology: Identifying Order in Complex Systems}
%
%\normalsize{(This meeting is unofficially partnered with ATiA and meets via a different link.)}
%\end{center}

\begin{abstract}
Bifurcations in a dynamical system are drastic behavioral changes, thus being able to detect the parameter values for which these bifurcations occur is essential to understanding the system overall. We develop a one-step method to study and detect bifurcations using zigzag persistent homology. While standard persistent homology has been used in this setting, it usually requires analyzing a collection of persistence diagrams, which in turn drives up the computational cost. Using zigzag persistence, we can capture topological changes in the state space of the dynamical system in only one persistence diagram.
\end{abstract}

\end{document}
