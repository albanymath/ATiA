% !TeX program = xelatex
\documentclass{UAmathtalk}

\author{Joe Moeller}
\address{UC Riverside \& NIST}
\urladdr{https://joemathjoe.wordpress.com/}
\title{Network Operads from Monoidal Species}
\date{Friday, April 23th, 2021}


\begin{document}

\maketitle

\begin{abstract}
Networks can be combined in various ways, such as overlaying one on top of another or setting two side by side. We introduce "network models", a monoidal variant of Joyal's combinatorial species, to encode these ways of combining networks. Different network models describe different kinds of networks. We show that each network model gives rise to an operad, whose operations are ways of assembling a network of the given kind from smaller parts. Such operads, and their algebras, can serve as tools for designing networks. The construction of the corresponding operad proceeds via a symmetric monoidal version of the Grothendieck construction.
\end{abstract}

\end{document}
