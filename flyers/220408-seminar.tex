% !TeX program = xelatex
\documentclass{UAmathtalk}

\author{Gillian Grindstaff}
\address{Oxford}
\urladdr{https://www1.maths.ox.ac.uk/people/gillian.grindstaff?migrdr=1}
\title{Generalizing geometric comparison of phylogenetic trees}
\date{Friday, April 8, 2022}


\begin{document}

\maketitle

\begin{abstract}
The study of phylogenetic trees has been motivated for decades by applications in evolutionary biology; it represents a quantitative historical branching relationship between a set of species, given as leaves of an evolutionary tree. More generally, phylogenetic trees are weighted, leaf-labeled trees realizing a particular class of acyclic finite metric spaces, encoding either a branching, merging, or static curvature relationship on a unique base set. Billera, Holmes, and Vogtmann define a metric on the moduli space of phylogenetic trees to form BHV space (2001), which provides a natural geometric setting for describing point clouds of different tree shapes on the same set of species. Using a combinatorial construction of BHV space as a piecewise-Euclidean cube complex with CAT(0) curvature, we can extend Euclidean statistics to phylogenetic tree space, providing efficient and unique Fr\'{e}chet means and confidence regions. However, it does not allow for the comparison of trees on nonidentical species sets (i.e., with different numbers of leaves), and in this context it is not evident how to relate tree spaces of different dimension. Using mainly combinatorics and linear optimization, we give a full characterization of the subspace of extensions of a subtree, computing the inverse of the ``forgetful'' map and allowing for supertree construction, which joins metric subtrees into a single space. We define a space of possible supertrees, optimize over it, and, for a collection of tree fragments with different leaf sets, quantify their compatibility.
\end{abstract}

\end{document}
