% !TeX program = xelatex
\documentclass{UAmathtalk}

\author{Hans Reiss}
\address{University of Pennsylvania}
\urladdr{https://hans-riess.github.io/}
\title{Tarski Sheaves}
\date{Friday, February 12, 2021}


\begin{document}

\maketitle

\begin{abstract}
In this talk, I review recent collaborative work with Robert Ghrist on laying the foundation for a homological Hodge theory of cellular sheaves of lattices, rightfully dubbed ``Tarski sheaves.'' 
This work is motivated by recent efforts by Ghrist and Hansen to lift spectral graph theory to sheaves, capitalizing on the Hodge decomposition of the cochain complex of a familiar cellular sheaf (of inner-product spaces). 
One goal of my research is to study more familiar sheaves such as sheaves of vector spaces or sheaves of sets by categorifying them. 
Lattice theory is a tactile playground for doing category theory with an eye towards computation. 
I will present a lattice-theoretic analog of the Hodge Theorem which relates the fixed points of the Laplacian of a Tarski Sheaf: the Tarski Laplacian. 
This main result---and its issuing cohomology theory---relies heavily on the famous folk theorem often attributed to Tarski, the Tarski Fixed Point Theorem. 
Along the way, I will provide plenty of examples. 
Time permitting, I will share recent progress between myself, Ghrist and North on further categorifying the Hodge Theorem into a statement about sheaves of categories and their adjunctions.
\end{abstract}

\end{document}
