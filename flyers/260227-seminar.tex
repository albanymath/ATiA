% !TeX program = xelatex
\documentclass{UAmathtalk}
\renewcommand{\when}{12:00 p.m.}
\renewcommand{\where}{Hudson 0110}
\author{Jānis Lazovskis}
\address{University of Latvia}
\urladdr{https://jlazovskis.com/}
\title{Topological methods for computational ecology}
\date{Friday, February 27, 2026}


\begin{document}

\maketitle

\begin{abstract}
Ecologists interested in the large-scale properties of a particular species often consider the niche, a high dimensional space of environmental variables which are suitable (and have been observed) for the species. Issues of sampling bias, connectedness and convexity assumptions, as well as computational requirements for inferential statistics complicate the translation from raw data to ecological claims. We are developing a method that aims to clarify some of these complications, in the sense of giving precise bounds for how the persistent homology changes when transforming irregular, large data sets into something that reflects more the heuristic expectation for the "shape of a species." Specifically, we give bounds in the bottleneck distance for how the dimension 0 persistence diagram changes when adding points by barycentric subdivision of the Vietoris-Rips complex, sparsifying by taking landmarks, and aligning points to a grid by floor division. Aligning to a grid also allows us to describe the complement of the data set, which gives information about codimension 1 homology by duality. This is joint work with Ran Levi and Juliano Morimoto, and is supported by the Latvian Council of Science grant 1.1.1.9/LZP/1/24/125.
\end{abstract}
\end{document}
