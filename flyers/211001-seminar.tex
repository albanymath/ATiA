% !TeX program = xelatex
\documentclass{UAmathtalk}

\author{Ad\'{e}lie Garin}
\address{EPFL}
\urladdr{https://www.epfl.ch/labs/hessbellwald-lab/members/adelie-garin/}
\title{From Trees to Barcodes and Back Again:\\Combinatorial and Probabilistic Perspectives}
\date{Friday, October 1st, 2021}


\begin{document}

\maketitle

%\begin{center}
%\Large{Workshop in Topology: Identifying Order in Complex Systems}
%
%\normalsize{(This meeting is unofficially partnered with ATiA and meets via a different link.)}
%\end{center}

\begin{abstract}
Methods of topological data analysis have been successfully applied in a wide range of fields to provide useful summaries of the structure of complex data sets in terms of topological descriptors, such as persistence diagrams, or barcodes. While there are many powerful techniques for computing topological descriptors, the inverse problem, i.e., recovering the input data from topological descriptors, has proved to be challenging. In this talk, I will consider specifically the inverse problem from barcodes to merge trees. I will describe a connection between the space of barcodes and symmetric groups, and show how to use it to study distributions of neurons modelled as trees, creating a bridge between the field of permutation statistics and TDA. I will then extend this symmetric group connection into a new way to coordinatize the space of barcodes, opening the door to a statistical and probabilistic study of the space of barcodes using a geometric group theory point of view. 
This is joint work with B. Brück, J. Curry, J. DeSha, K. Hess, L. Kanari and B. Mallery.
\end{abstract}

\end{document}
