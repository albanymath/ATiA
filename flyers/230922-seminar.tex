%!TeX program = xelatex
\documentclass{UAmathtalk}

\author{Sarah Percival}
\address{MSU}
\urladdr{https://sperciva.github.io/}
\title{Bounding the Interleaving Distance for Geometric Graphs with a Loss Function}
\date{Friday, September 22, 2023}


\begin{document}

\maketitle

\begin{abstract}
Data consisting of a graph with a function to $\mathbb{R}^d$ arise in many applications, encompassing structures such as Mapper graphs, Reeb graphs, geometric graphs, and 3D skeletons. 
As such, the ability to compare and cluster such objects is required in a data analysis pipeline, leading to a need for distances or metrics on these objects.  
In this work, we study the interleaving distance on discretizations of these objects, where functor representations of data can be compared by finding pairs of natural transformations between them. 
However, in many cases, particularly those of the set-valued functor variety, computation of the interleaving distance is NP-hard. 
For this reason, we take inspiration from the work of Robinson (2020) to find quality measures for families of maps that do not rise to the level of a natural transformation, called assignments. 
We then endow the functor images with the extra structure of a metric space and define a loss function which measures how far a pair of assignments are from making the required diagrams of an interleaving commute. 
Finally we show that the computation of the loss function is polynomial in several use cases of interest. 
This is joint work with Erin Chambers, Elizabeth Munch, and Bei Wang.
\end{abstract}

\end{document}
