% !TeX program = xelatex
\documentclass{UAmathtalk}

\author{Francesca Tombari}
\address{KTH}
\urladdr{https://www.kth.se/profile/tombari?l=en}
\title{Realisations of Posets and Tameness}
\date{Friday, February 25, 2022}


\begin{document}

\maketitle

\begin{abstract}
The central theme of this talk will be a new construction called \textbf{realisation}, which transforms posets into posets. Intuitively the realisation associates a continuous structure to a locally discrete poset by filling in empty spaces. For example, it transforms a discrete grid to a continuous one. 

It is known that upper semilattices are suitable posets to talk about persistence. Although different, realisations and upper semilattices share many key features, which explains why the former also fit well with persistence. This is unexpected as for example the realisation of an upper semilattice may fail to be an upper semilattice.

Realisations have well behaved and explicit discrete approximations, which are essential in persistence and to describe tame properties of functors. For this reason, they form a rich class of examples of indexing posets for persistence modules, containing many familiar ones, such as the k-dimensional real spaces and the zigzag posets.

Similarities between realisations and upper semilattices span over an even wider range of properties. For example, dimensions of points, a combinatorial notions we will introduce, in realisations behave in a similar way as they do in upper semilattices. They also share key homological properties. For instance the homological dimension (i.e., the length of the minimal free resolution) and Betti numbers of tame vector space valued functors indexed by both realisations and upper semilattices can be calculated using Koszul complexes. Consequently, calculating these invariants for functors indexed by both types of posets can be done directly, avoiding constructing explicit resolutions. 

This is joint work with Wojciech Chach\'{o}lski and Alvin Jin.
\end{abstract}

\end{document}
