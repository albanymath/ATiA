% !TeX program = xelatex
\documentclass{UAmathtalk}

\author{Barbara Giunti}
\address{TU Graz}
\urladdr{https://www.bgiunti.info/}
\title{Amplitudes in Multiparameter Persistence}
\date{Friday, March 11, 2022}


\begin{document}

\maketitle

\begin{abstract}
In persistence theory, one starts with data and then build a parameterized object from them, which is then analysed via some invariants, typically retrieved using homology. This pipeline is stable in 1-parameter persistence, where the invariants are all given by the barcode. Since there is no proper generalization of barcode in the multiparamenter case, the choice of the invariants - which is strictly related to the choice of the distances used to ensure the stability of the persistence pipeline - depends on the specific information one wants to extract from the data. Different invariants may greatly vary their behaviour and be stable only under very different distances.
With the goal of addressing this problem, we introduce amplitudes, which are invariants that arise from assigning a non-negative real number to each persistence module, and are monotone and subadditive in an appropriate sense. There are different ways to associate a distance to an amplitude, which is useful in practical applications. Indeed, this gives us the possibility to define metrics based on the wanted invariants and fine-tune the stability results. Our framework is very comprehensive, as many different invariants that have been introduced in the Topological Data Analysis literature are examples of amplitudes. Furthermore, many known distances for multiparameter persistence can be shown to be distances from amplitudes. In addition, this framework allows us to compare different distances and can be used to prove new stability results.

\end{abstract}

\end{document}
