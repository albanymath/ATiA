% !TeX program = xelatex
\documentclass{UAmathtalk}

\author{Sarah Percival}
\address{Michigan State University}
\urladdr{https://bmb.natsci.msu.edu/postdocs/sarah-percival/}
\title{An Efficient Algorithm for the Computation of Reeb Graphs from Roadmaps}
\date{Friday, November 12, 2021}


\begin{document}

\maketitle


\begin{abstract}
The Reeb graph, a tool from Morse theory, has recently found use in applied topology due to its ability to track the changes in connectivity of level sets of a function. The roadmap of a set, a construction that arises in semi-algebraic geometry, is a one-dimensional set that encodes information about the connected components of a set. In this talk, I will show that the Reeb graph and, more generally, the Reeb space, of a semi-algebraic set is homeomorphic to a semi-algebraic set, which raises the algorithmic problem of computing a semi-algebraic description of the Reeb graph. We present an algorithm with singly-exponential complexity that realizes the Reeb graph of a function $f:X \to Y$ as a semi-algebraic quotient using the roadmap of $X$ with respect to $f$.
\end{abstract}

\end{document}
