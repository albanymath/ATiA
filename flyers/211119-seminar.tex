% !TeX program = xelatex
\documentclass{UAmathtalk}

\author{Darrick Lee}
\address{EPFL}
\urladdr{https://www2.math.upenn.edu/~ldarrick/}
\title{Signatures, Lipschitz-free Spaces, and Paths of Persistence Diagrams}
\date{Friday, November 19, 2021}


\begin{document}

\maketitle


\begin{abstract}
Paths of persistence diagrams describe the temporally evolving topology of dynamic point cloud data sets. As with the case of static diagrams, such objects are difficult to work with in a machine learning setting, and a feature map is often required. The path signature provides a reparametrization-invariant feature map which is both universal and characteristic, allowing us to study both functions and measures on the space of paths of persistence diagrams via kernel methods. We explore the theoretical and computational aspects of using path signatures to study such paths, and apply it to a parameter estimation problem for models of collective motion.
\end{abstract}

\end{document}
