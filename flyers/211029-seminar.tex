% !TeX program = xelatex
\documentclass{UAmathtalk}

\author{Yohai Reani}
\address{Technion}
\urladdr{https://il.linkedin.com/in/yohai-reani-ba410378}
\title{The Coupled Alpha Complex }
\date{Friday, October 29th, 2021}


\begin{document}

\maketitle

\begin{center}
\Large{{\bf UNUSUAL TIME:} 9:00 am EST / 4:00 pm IST}
\end{center}

\begin{abstract}
The alpha complex is a subset of the Delaunay triangulation and is often used in computational geometry and topology. One of the main drawbacks of using the alpha complex is that it is non-monotone, in the sense that the alpha complex of a given set of points does not necessarily (and generically not) include the alpha complex generated by a subset of these points. The lack of monotonicity may introduce significant computational costs when using the alpha complex, and in some cases even render it unusable. In this talk we will present a new construction based on the alpha complex, that is homotopy equivalent to the alpha complex while maintaining monotonicity, and is not significantly more costly to use than the standard alpha complex. We will demonstrate the power of this complex through the application of cycle registration in persistent homology. This work is a joint work with Omer Bobrowski, Technion. 
\end{abstract}

\end{document}
